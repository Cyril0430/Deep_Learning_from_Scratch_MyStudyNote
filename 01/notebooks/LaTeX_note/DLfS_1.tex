\documentclass[fontsize = 10.5pt]{jlreq}
\usepackage{mymathnote}

\begin{document}

\section{Softmax-with-Loss レイヤを追う}

Softmax関数への入力を $\bm{X}: (N, m)$ と仮定する。(バッチサイズが $N$、各データに $m$ 個の値が格納)。

\[
\bm{Y} = \text{softmax}(\bm{X}) \implies \bm{Y}: (N, m)
\]

教師データ $\bm{T}$ の形状は $(N, m)$ となる必要があるので、交差エントロピー誤差 $L$ は以下のようになる。

\[
L = \text{cross\_entropy\_error}(\bm{Y}, \bm{T})
\]

\subsection*{One-hot vector の $\bm{T}$}
\[
\bm{T} = \mqty(t_{11} & \cdots & t_{1m} \\ \vdots & \ddots & \vdots \\ t_{N1} & \cdots & t_{Nm})
\]
\[
(\bm{T})_{ij} = 
\begin{cases} 
1 & (i \text{番目のデータの正解が} j-1 \text{のとき}) \\
0 & (\text{otherwise})
\end{cases}
\]

\subsection*{実装上の注意:`cross\_entropy\_error` 関数の内部}


% \begin{lstlisting}[language=Python]
% if t.size == y.size:
%     t = t.argmax(axis=1)
% \end{lstlisting}

\begin{mynote}{疑問:なぜ index に変換するのか?}
これによって $\bm{t}$ が one-hot から正解ラベルのインデックスになるのはなぜか?

$\Rightarrow$ そもそも、まず $\bm{t}$ と $\bm{y}$ の形状が等しいときは、$\bm{t}$ が one-hot である可能性がある($\bm{y}$ が $(1, m)$ という形状のときは、$\bm{t}$ がどちらか分からないが)。

仮に $\bm{y}: (N, m)$ とする。そして、$\bm{t}$ が one-hot のときは、必ず形状が $(N, m)$ で $\bm{y}$ と同じになる。
これによって \texttt{if t.size == y.size} というように判別できる。

\texttt{t = t.argmax(axis=1)} とすると、$\bm{T} (= \bm{t})$ は $(N, m)$ から $(N, )$ という行ベクトルになる。そしてその各要素には、正解ラベルのインデックス(1が格納されている要素番号)が入る。
\end{mynote}

$\bm{T}$ の形状が $\bm{Y}$ の形状と等しいとき、$\bm{T}$ は one-hot vector であるとみなして処理を行う。

Returnされるのは、
\[
\frac{1}{N} \sum_{k} E_k \quad (\text{ただし } E_k = - \sum_{k} t_{nk} \log y_{nk})
\]
つまり、スカラー(次元は $0$)が返る。

\subsection*{逆伝播 (Backpropagation)}

\[
\text{d}\bm{x} = (\text{self.y} - \text{self.t}) \mathop{/} \text{batch\_size}
\]

\begin{mynote}{なぜバッチサイズで割る?}
今採用している交差エントロピー誤差は
\[
L = - \frac{1}{N} \sum_{i=1}^N \sum_{k=1}^m t_{ik} \log y_{ik}
\]
である。そうすると、前に求めた $\pdv{L}{y_{ij}}$ 等を、この $L$ の場合で再度求めると、
\[
\pdv{L}{a_{ij}} = \frac{1}{N} (y_{ij} - t_{ij})
\]
になるため。
\end{mynote}

\newpage

\section{学習に関するテクニック (§6)}

これまでは確率的勾配降下法 (SGD) を用いてきた。
\[
\bm{W} \leftarrow \bm{W} - \eta \pdv{L}{\bm{W}}
\]
However, there are some problems. Those are below.

\subsection*{SGDのジグザグ問題}

ここから下る時、必ずしも目指す所(最小値)へと一直線になるとは限らない。
関数の形状が異方性(anisotropic)を持つ場合、勾配の方向が最小値を指さないことがある。

\begin{center}
\begin{tikzpicture}
    % 簡易的な等高線の描画
    \draw[gray, thin] (0,0) ellipse (2 and 1);
    \draw[gray, thin] (0,0) ellipse (1.5 and 0.75);
    \draw[gray, thin] (0,0) ellipse (1 and 0.5);
    \draw[fill=black] (0,0) circle (2pt) node[right] {目指す所};
    
    % 軸
    \draw[->] (-2.5, -1.5) -- (2.5, -1.5);
    \draw[->] (-2.5, -1.5) -- (-2.5, 1.5);

    % ジグザグパス
    \draw[->, red, thick] (-2, 1) -- (-1.5, -0.5);
    \draw[->, red, thick] (-1.5, -0.5) -- (-1, 0.8);
    \draw[->, red, thick] (-1, 0.8) -- (-0.5, -0.2);
    \draw[->, red, thick] (-0.5, -0.2) -- (0, 0);
    
    \node[red, font=\small] at (-2, 1.3) {ここから下るとき};
    \node[blue, font=\small] at (-0.5, -0.8) {このように進むかもしれない(非効率)};
\end{tikzpicture}
\end{center}

解決策として次の3つがある。

\subsection{Momentum (モーメンタム)}
物理法則を借りる手法。
速度 (velocity) $\bm{v}$ という変数を導入。
$\Rightarrow$ 今まで動いてきた方向(勢い)を保ち(慣性)、新しい勾配の方向に力を加える。

\begin{mynote}{AIからの質問 (Memo)}
活性化関数を ReLU から Sigmoid にすると、層が深くなっていくと逆伝播な勾配はどうなっていくか?
$\rightarrow$ Vanished (勾配消失)。パラパラと第6章を読んでいくのが良さそう。
\end{mynote}

\newpage

\section{Batch Normalization}

Batch Norm レイヤは、$\hat{x}_i \ (i=1, \dots, m)$ に対し、次のような変換をする。

\begin{center}
\begin{tikzpicture}[node distance=1.5cm, auto,
    block/.style={rectangle, draw, minimum height=1cm, minimum width=1.5cm},
    line/.style={-Stealth}]

    \node (in) {入力};
    \node[block, right=of in] (affine1) {Affine};
    \node[block, right=of affine1, draw=red, thick, align=center] (bn) {Batch\\Norm};
    \node[block, right=of bn] (relu) {ReLU};
    
    \path[line] (in) -- (affine1);
    \path[line] (affine1) -- (bn);
    \path[line] (bn) -- (relu);
    
    % Note
    \node[draw=red, circle, minimum size=3cm] at (bn) {};
    \node[above=0.5cm of bn, red] {データの分布を正規化するレイヤを挿入};
\end{tikzpicture}
\end{center}

\subsection*{順伝播 (Forward)}
学習を行う際、ミニバッチを単位として、ミニバッチ毎に正規化を行う。
i.e. データの分布について、平均が $0$、分散が $1$ になるようにする。

\begin{align}
\mu_B &\leftarrow \frac{1}{m} \sum_{i=1}^m x_i \\
\sigma_B^2 &\leftarrow \frac{1}{m} \sum_{i=1}^m (x_i - \mu_B)^2 \\
\hat{x}_i &\leftarrow \frac{x_i - \mu_B}{\sqrt{\sigma_B^2 + \epsilon}} \quad (\epsilon \text{は0除算防止の微小値}) \\
y_i &\leftarrow \gamma \hat{x}_i + \beta
\end{align}

$\gamma, \beta$ はパラメータ。学習により調整されていく(初期値:$\gamma=1, \beta=0$)。
これだけだと、単なる標準化のみになってしまうため、固有のスケール($\gamma$)とシフト($\beta$)を与える。

\subsection*{逆伝播 (Backward)}

微分の連鎖律を用いて導出する。

\[
\pdv{L}{\hat{x}_i} = \pdv{L}{y_i} \cdot \pdv{y_i}{\hat{x}_i} = \pdv{L}{y_i} \cdot \gamma
\]

ここからが複雑である。$x_1 \sim x_m$ のすべてが各 $\hat{x}_i$ に影響を与えているため、和をとる必要がある箇所に注意。

\begin{definition}{分散に関する微分}
\[
\pdv{\sigma_B^2}{x_i} = \frac{2}{m} (x_i - \mu_B)
\]
※ ノート内の取り消し線部修正:平均 $\mu_B$ も $x_i$ の関数であるため、厳密にはもう少し複雑だが、ここでは簡略化して記述されている可能性あり。
\end{definition}

具体的な逆伝播のステップ(ノートの計算過程):

\begin{align}
\pdv{\hat{x}_i}{\sigma_B^2} &= (x_i - \mu_B) \cdot \left( -\frac{1}{2} (\sigma_B^2 + \epsilon)^{-3/2} \right) \\
&= - \frac{x_i - \mu_B}{2 (\sigma_B^2 + \epsilon)\sqrt{\sigma_B^2 + \epsilon}}
\end{align}

\begin{align}
\pdv{L}{\sigma_B^2} &= \sum_{i=1}^m \pdv{L}{\hat{x}_i} \pdv{\hat{x}_i}{\sigma_B^2} \\
&= - \sum_{i=1}^m \left( \pdv{L}{\hat{x}_i} \cdot \frac{x_i - \mu_B}{2(\sigma_B^2 + \epsilon)^{1.5}} \right)
\end{align}

また、平均 $\mu_B$ に関する微分は、$\hat{x}_i$ からのルートと、$\sigma_B^2$ からのルートの2つがあることに注意。

\[
\pdv{L}{\mu_B} = \sum_{i=1}^m \pdv{L}{\hat{x}_i} \pdv{\hat{x}_i}{\mu_B} + \pdv{L}{\sigma_B^2} \pdv{\sigma_B^2}{\mu_B}
\]

最終的に $\pdv{L}{x_i}$ を求めるには、上記の各項を統合する。

\begin{mynote}{計算グラフによる理解}
計算グラフを書くことで、何が何にどのように影響を与えているかが見えてくる。
順方向:分散 $\Rightarrow$ 集計。
逆方向:分配(コピー)。
\end{mynote}

\newpage

\section{畳み込みニューラルネットワーク (CNN) (§7)}

\subsection*{7.1 全体の構造}

これまでに見たネットワーク(全結合層):
\begin{center}
    \texttt{Affine $\to$ ReLU $\to$ Affine $\to$ ReLU $\to$ ...}
\end{center}

\textbf{CNN (Convolutional Neural Network):}
新出レイヤとして「Convolution Layer (畳み込み層)」と「Pooling Layer (プーリング層)」が登場する。

どのようにレイヤを組み合わせるか?

\begin{center}
\begin{tikzpicture}[
    node distance=0.8cm,
    layer/.style={rectangle, draw, minimum height=1.5cm, minimum width=1cm, align=center},
    conv/.style={layer, pattern=north east lines, pattern color=red!30},
    pool/.style={layer, pattern=dots, pattern color=blue!30}
    ]

    \node[layer] (input) {In};
    \node[conv, right=of input] (conv1) {Conv};
    \node[layer, right=of conv1] (relu1) {ReLU};
    \node[pool, right=of relu1] (pool1) {Pool};
    
    \node[right=of pool1] (dots) {$\cdots$};
    
    \node[conv, right=of dots] (conv2) {Conv};
    \node[layer, right=of conv2] (relu2) {ReLU};
    \node[layer, right=of relu2] (affine) {Affine};
    \node[layer, right=of affine] (relu3) {ReLU};
    
    \draw[->] (input) -- (conv1);
    \draw[->] (conv1) -- (relu1);
    \draw[->] (relu1) -- (pool1);
    \draw[->] (pool1) -- (dots);
    \draw[->] (dots) -- (conv2);
    \draw[->] (conv2) -- (relu2);
    \draw[->] (relu2) -- (affine); % ここから途中(全結合)に戻るイメージ
    \draw[->] (affine) -- (relu3);

\end{tikzpicture}
\end{center}

出力に近い層では、これまでの \texttt{Affine $\to$ ReLU} の形に戻るのが一般的。
\end{document}