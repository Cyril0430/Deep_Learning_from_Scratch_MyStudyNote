\documentclass[
  dvipdfmx,
  fontsize = 10.5pt
]{jlreq}
\usepackage{style/note}

\title{ゼロから作る Deep Learningの数理的背景}
% \author{森口志龍}
\author{}
\date{最終更新日:\today}

\begin{document}
\maketitle
\begin{abstract}
  この資料は私が気まぐれで作ったものなので、続くかどうかは分かりません。

  % この資料の目的は、斎藤康毅の著書「ゼロから作るDeep Learning」の数理的背景をまとめた資料である。本著書には、数理的背景が厳密に議論されていない部分がいくつか見られる。例えばAffineレイアを実装する際の各偏微分の計算過程が省略されているところが見受けられる。
  本資料では、「ゼロから作るDeep Learning」では語られていない数理的背景や式変形での行間を埋めることを目的とする。
\end{abstract}
\tableofcontents

\section{バッチ版Affineレイヤ実装}


\end{document}